% Template for the iniital assignment for the seminar 
% Domain-specific architectures for deep neural networks"
% summer semester 2020
%

\documentclass[10pt]{article}


\usepackage{url}
\usepackage[english]{babel}
\usepackage[T1]{fontenc}
\usepackage[utf8]{inputenc}
\usepackage{times}
\usepackage{geometry}
\usepackage{graphicx}
\usepackage{paralist}
\usepackage{xcolor}


\geometry{verbose,paperwidth=21cm,paperheight=29.7cm}
\geometry{tmargin=20mm,bmargin=25mm,lmargin=20mm,rmargin=20mm}



%%%%%%%%%%%%%%%%%%%%%%%%%%%%%%%%%%%%%%%%%%%%%%%%%%%%%%%%%%%%%%%%%%%%%%%%%%%%%%
%%%%%%%%% configure these settings and you are good to go %%%%%%%%%%%%%%%%%%%%
%%%%%%%%%%%%%%%%%%%%%%%%%%%%%%%%%%%%%%%%%%%%%%%%%%%%%%%%%%%%%%%%%%%%%%%%%%%%%%
\newcommand{\participant}{Sachin Sudhakar Shetty}
\newcommand{\affiliation}{Paderborn University}
%\urldef{\emailaddress}\url{<my-UPB-email-address>}
\newcommand{\topic}{Assignment 5: Group Project Interactivity}
\newcommand{\submissiondate}{June 17th, 2020}


\begin{document}

\title{\topic}
\author{\Large{\participant}\\ \affiliation \\ {\small \emailaddress}}
\date{\submissiondate}
\maketitle
\thispagestyle{empty}


%%%%%%%%%%%%%%%%%%%%%%%%%%%%%%%%%%%%%%%%%%%%%%%%%%%%%%%%%%%%%%%%%%%%%%%%%%%%%%
%%%%%%%%% here starts the actual text                     %%%%%%%%%%%%%%%%%%%%
%%%%%%%%%%%%%%%%%%%%%%%%%%%%%%%%%%%%%%%%%%%%%%%%%%%%%%%%%%%%%%%%%%%%%%%%%%%%%%


\section{Introduction}
%\label{sec:introduction}

\color{black}
At present, where University, Organisation, Institution, Restaurant etc... Completely Shut but except Hospitals which are working 24*7, yes its {\bf COVID-19}. A pandemic which has shaken the economy of the world. As battle continues, data sets corresponding to the virus disease is collected around the globe. This concept  paper is on the Topic: COVID-19 as part of the Assignment  to the subject Module: Interactive Data Visualisation which we here summaries about the DataSet of covid-19 disease used and visualisation plus interaction provided to the end user.	
\color{black}

\section{DataSet}
Ramu
\section{User and Task}
The owid-covid-19-data DataSet captures information of disease spread  and fatalities caused around the globe and also counter measure taken by respective countries.\newline
Tasks for the Visualisation can be as follows:
\begin{itemize}
    \item Visualisation of the graph of disease information country-wise Infected people, Fatality, Number of Test Conducted.
    \item Visualisation by comparison between the medical infrastructure of the country, fatality count, cases reported.
    \item Visualisation of this multivariate DataSet taking into account the country, cases reported, fatality, gdp per captital, beds per million etc.
\end{itemize}
Visualisation targeted Users can be as follows:
\begin{itemize}
    \item Medical institution can use the visualisation to study curve pattern of the infection spread, specific to region and prepare stratergies for the future similar infection outbreak.
    \item Monetary Institution like: World Bank can estimate the impact of the losses of the particular region and aid respectively.
    \item Hospitals and Medical Institution can stratergies to priorities  medical attention to particular age group or people with chronic disease.
    \item Better planning medical infrastructure by government body for the future days.
\end{itemize}}
\section{Visulization Technique}
Rajath & visvash
\section{Interaction}
Rahul
%\label{sec:First_Section}

....




%% the following commands include the biliographic information (in BibTeX format) from the 
%% file template.bib
%\bibliography{Template_One_Pager.bib}
%\bibliographystyle{plain}

\end{document}
