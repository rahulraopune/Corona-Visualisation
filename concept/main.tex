% 
\documentclass[10pt]{article}



\usepackage{url}
\usepackage[english]{babel}
\usepackage[T1]{fontenc}
\usepackage[utf8]{inputenc}
\usepackage{times}
\usepackage{geometry}
\usepackage{graphicx}
\usepackage{paralist}
\usepackage{xcolor}
% \usepackage{fancyhdr}

\geometry{verbose,paperwidth=21cm,paperheight=29.7cm}
\geometry{tmargin=20mm,bmargin=25mm,lmargin=20mm,rmargin=20mm}

% \fancyhead{}
% \fancyhead[LO]{Team Mask (Group 09): Rahul Rao, Rajath Acharya, Ramu Ramasamy, Sachin Sudhakar Shetty, Vishwas}
% \fancyfoot[LO]{Interactive Data Visualization}
% \fancyfoot[E]{\thepage}
% \fancyfoot[R]{\thepage}
% \fancyfoot[C]{}

%%%%%%%%%%%%%%%%%%%%%%%%%%%%%%%%%%%%%%%%%%%%%%%%%%%%%%%%%%%%%%%%%%%%%%%%%%%%%%
%%%%%%%%% configure these settings and you are good to go %%%%%%%%%%%%%%%%%%%%
%%%%%%%%%%%%%%%%%%%%%%%%%%%%%%%%%%%%%%%%%%%%%%%%%%%%%%%%%%%%%%%%%%%%%%%%%%%%%%
\newcommand{\participant}{Team Mask (Group 09)}
\newcommand{\affiliation}{Paderborn University}
% \urldef{\emailaddress}\url{<my-UPB-email-address>}
\newcommand{\topic}{Assignment 5: Group Project Interactivity}
\newcommand{\submissiondate}{June 17, 2020}

% 

\begin{document}

\title{\topic}
\author{\Large{\participant}\\ \affiliation \\}
\date{\submissiondate}
\maketitle
\thispagestyle{empty}


%%%%%%%%%%%%%%%%%%%%%%%%%%%%%%%%%%%%%%%%%%%%%%%%%%%%%%%%%%%%%%%%%%%%%%%%%%%%%%
%%%%%%%%% here starts the actual text                     %%%%%%%%%%%%%%%%%%%%
%%%%%%%%%%%%%%%%%%%%%%%%%%%%%%%%%%%%%%%%%%%%%%%%%%%%%%%%%%%%%%%%%%%%%%%%%%%%%%


\section{Introduction}
%\label{sec:introduction}

\color{black}
At present, where University, Organization, Institution, Restaurant etc... Completely shut but except Hospitals which are working 24*7, yes its {\bf COVID-19}. A pandemic which has shaken the economy of the world. As battle continues, data sets corresponding to the virus disease is collected around the globe. This concept  paper is on the Topic: COVID-19 as part of the Assignment  to the subject Module: Interactive Data Visualization which we here summaries about the DataSet of covid-19 disease used and visualization plus interaction provided to the end user.	
\color{black}

\section{Data Characteristics}

The dataset characteristics are described in the Table \ref{tab:data_chars}. The dataset comprises data from different sources mentioned in Table \ref{tab:data_sources}. All the data are comprised in the file \textit{owid-covid-data.csv} and the data is available as comma separated strings. The dataset contains 22873 records and 33 columns (on June 12, 2020). Also, dataset containing the polygons, used to form a World Map for representing geo spatial mapping information is sourced from Natural Earth (Admin-0 Countries). 

% that include European Centre for Disease Prevention and Control, National government reports, United Nations, Department of Economic and Social Affairs, Population Division, World Population Prospects: The 2019 Revision, World Bank – World Development Indicators, sourced from Food and Agriculture Organization and World Bank estimates, UN Population Division, World Population Prospects, 2017 Revision, United Nations Statistics Division, and OECD, Eurostat, World Bank, national government records and other sources.

\begin{table}[htbp]
    \centering
    \begin{tabular}{|p{6cm}|p{6cm}|}
        \hline
        European Centre for Disease Prevention and Control &  National government reports \\
        \hline 
        Department of Economic and Social Affairs & United Nations \\
        \hline
        World Population Prospects: The 2019 Revision & Population Division \\
        \hline 
        OECD & Eurostat \\
        \hline
        World Bank – World Development Indicators, sourced from Food and Agriculture Organization and World Bank estimates &  UN Population Division, World Population Prospects, 2017 Revision\\
        \hline
    \end{tabular}
	\caption{Sources}
	\label{tab:data_sources}
\end{table}

\begin{table}[htbp]
    \centering
	\begin{tabular}{|l|l|p{8cm}|}
		\hline
		\textbf{Data} & \textbf{Data Type} & \textbf{Description} \\
        \hline \hline
        iso\_code & String & ISO 3166-1 alpha-3 (3 letter country codes) \\
        \hline
        continent & String & Continent name \\
        \hline
        location & String & Country name \\
        \hline
        date & String & Date of observation \\
        \hline
        total\_cases & int64 & Total confirmed cases of COVID-19 \\
        \hline
        new\_cases & int64 & New confirmed cases  \\
        \hline
        total\_deaths & int64 & Total deaths attributed \\
        \hline
        new\_deaths & int64 & New deaths \\
        \hline
        total\_cases\_per\_million & float64 & Total confirmed cases per one million people \\
        \hline
        new\_cases\_per\_million & float64 & New confirmed cases per one million people  \\
        \hline
        total\_deaths\_per\_million & float64 & Total deaths per one million people \\
        \hline
        new\_deaths\_per\_million & float64 & New deaths per one million people \\
        \hline
        total\_tests & float64 & Total tests conducted \\
        \hline
        new\_tests & float64 & New tests conducted \\
        \hline
        new\_tests\_smoothed & float64 & New tests (7 day smoothed). For countries not reporting data on daily basis. \\
        \hline
        new\_tests\_per\_thousand & float64 & New tests per thousand people \\
        \hline
        new\_tests\_smoothed\_per\_thousand & float64 & New tests per thousand people (7 day smoothed) \\
        \hline
        tests\_units & float64 & Units used by the location to report the testing data \\
        \hline
        population & float64 & Population in 2020 \\
        \hline
        population\_density & float64 & Number of people divided by land area, measured in square kilometers, most recent year available. \\
        \hline
        median\_age & float64 & Median age of the population, UN projection for 2020. \\
        \hline
        aged\_65\_older & float64 & Share of the population that is 65 years and older. \\
        \hline
        aged\_70\_older & float64 & Share of the population that is 70 years and older in 2015. \\
        \hline
        gdp\_per\_capita & float64 & Gross domestic product at purchasing power parity (constant 2011 international dollars), most recent year available. \\
        \hline
        extreme\_poverty & float64 & Share of the population living in extreme poverty, most recent year available since 2010. \\
        \hline
        cvd\_death\_rate & float64 & Death rate from cardiovascular disease in 2017. \\
        \hline
        diabetes\_prevalence & float64 & Diabetes prevalence (\% of population aged 20 to 79) in 2017. \\
        \hline
        female\_smokers & float64 & Share of women who smoke, most recent year available. \\
        \hline
        male\_smokers & float64 & Share of men who smoke, most recent year available. \\
        \hline
        handwashing\_facilities & float64 & Share of the population with basic handwashing facilities on premises, most recent year available. \\
        \hline
        hospital\_beds\_per\_thousand & float64 & Hospital beds per 1,000 people, most recent year available since 2010. \\
        \hline
    \end{tabular}
	\caption{Data Characteristics}
	\label{tab:data_chars}
\end{table}



\section{User and Task}
The owid-covid-19-data DataSet captures information of disease spread  and fatalities caused around the globe and also counter measure taken by respective countries.\newline
Tasks for the Visualisation can be as follows:
\begin{itemize}
    \item Visualisation of the graph of disease information country-wise Infected people, Fatality, Number of Test Conducted.
    \item Visualisation by comparison between the medical infrastructure of the country, fatality count, cases reported.
    \item Visualisation of this multivariate DataSet taking into account the country, cases reported, fatality, gdp per captital, beds per million etc.
\end{itemize}
Visualisation targeted Users can be as follows:
\begin{itemize}
    \item Medical institution can use the visualisation to study curve pattern of the infection spread, specific to region and prepare stratergies for the future similar infection outbreak.
    \item Monetary Institution like: World Bank can estimate the impact of the losses of the particular region and aid respectively.
    \item Hospitals and Medical Institution can stratergies to priorities  medical attention to particular age group or people with chronic disease.
    \item Better planning medical infrastructure by government body for the future days.
\end{itemize}


\section{Visulization Technique}
\begin{itemize}
    \item \textbf{Stacked Bar Chart:} Combining more then bar chart on top of each other for expanded viewing and visualisation, we make use of Stacked Bar Charts. Here, we take the population of the top 10 most affected countries in millions and then report the number of cases per million in each of these countries by Stacked Bar charts. We can also develop a strategy to visualise the number of hospital beds available to the number of beds occupied for these countries.
    \item \textbf{Parallel Coordinates:} The visualisation of the dataset can be done by grouping the top 10 affected countries together and then compare against each other, the various factors like deaths, active cases, hospital beds etc. to get an idea of how each country fares in comparison with the others.
    \item \textbf{GeoSpatial Plotting using Chloropleth Maps:} The earth with the various countries or the geographical areas are visualised which shows the intensity of values on equidistant rectangular projection of world Map which ca use various visual variables like color, texture, brightness etc. In our case pertaining to COVID19 we are using a mapping of the logarithm of the total count to the respective country using the colour to mark the intensity of value.
    \item \textbf{Superimposed line plotting:} This type of plotting is generally used to show the single dimensional values i.e one value per data item with respect to change in time.In case of superimposed line plotting multiple variables are plotted together with respect to time and in our case of COVID19 we are plotting profile line of the daily logarithmic total counts the top 10 countries having maximum cases of COVID19 differentiated with different colours on Y axis so that its easier to visualise the rise and drop of different cases on a daily or a monthly basis on the X axis.
\end{itemize}


\section{Interaction}
\begin{itemize}
    \item \textbf{Navigation and Selection:} Medical Institutions/Scientists can use the click and drag, double clicking(zooming) feature, plus and minus keys to zoom to change the level of detail on the choropleth map. And further using the selection feature(hovering) by clicking on a desired region on the choropleth map projects a pop-up of the cases reported, fatality count.   
    \item \textbf{Searching and Filtering:} Users can benefit from the search feature by searching with specific keyword such as name of a region and utilize filtering by reducing the complexity of visualization of all COVID-19 infection data by setting constraints (checkboxes to show only fatality count or cases reported) which cannot be expressed using keywords and provide simple organized data showing only the relevant information required for the users.
    \item \textbf{Connection:}Users can much better understand the progression of COVID-19 situation in specific region, if linking one part of the visualization to another. In our case, we can select a specific region on the choropleth map and it should also display equivalent line graph showing only the COVID-19 infection data such as number of cases reported vs. fatality count and elaborate more details on the selected region
\end{itemize}
%\label{sec:First_Section}

....




%% the following commands include the biliographic information (in BibTeX format) from the 
%% file template.bib
%\bibliography{Template_One_Pager.bib}
%\bibliographystyle{plain}

\end{document}
